\section{Experimental Setup}
\subsection{Environment}
Our experiments were conducted on the following hardware:
\begin{itemize}
    \item CPU: Intel Core i7-9700K @ 3.60GHz
    \item RAM: 32GB DDR4 @ 3200MHz
    \item OS: Windows 10 Pro
\end{itemize}

\subsection{Implementation Details}
The BufferedArrayList implementation uses the following parameters:
\begin{itemize}
    \item CHUNK\_SIZE = 64 elements
    \item Initial capacity = 16 chunks
    \item Growth factor = 1.5
\end{itemize}

\subsection{Benchmark Methodology}
Our benchmarking approach includes:
\begin{itemize}
    \item Warm-up iterations: 5
    \item Measurement iterations: 5
    \item Fork count: 1
    \item Timeout: 10 minutes per benchmark
\end{itemize}

\subsection{Test Cases}
We evaluated the following scenarios:
\begin{itemize}
    \item Small dataset (100K elements)
    \item Medium dataset (1M elements)
    \item Various operation types
    \item Different access patterns
\end{itemize}

\begin{table}[t]
\centering
\begin{tabular}{|l|c|c|}
\hline
\textbf{Parameter} & \textbf{Value} & \textbf{Description} \\
\hline
CHUNK\_SIZE & 64 & Elements per chunk \\
\hline
INITIAL\_CAPACITY & 16 & Initial number of chunks \\
\hline
GROWTH\_FACTOR & 1.5 & Chunk array growth rate \\
\hline
BUFFER\_SIZE & 32 & Size of insertion buffer \\
\hline
\end{tabular}
\caption{Implementation parameters}
\label{tab:parameters}
\end{table} 